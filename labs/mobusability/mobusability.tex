\documentclass[a4paper,logo,nocourse]{miunasgn}
\usepackage[utf8]{inputenc}
\usepackage[T1]{fontenc}
\usepackage[english,swedish]{babel}
\usepackage[hyphens]{url}
\usepackage{hyperref}
\usepackage{prettyref,varioref}
\usepackage[today,nofancy]{svninfo}
\usepackage[natbib,style=authoryear-comp,maxbibnames=99]{biblatex}
\addbibresource{literature.bib}
\usepackage[varioref,prettyref]{miunmisc}

\svnInfo $Id$
%\printanswers

\courseid{IG025G}
\course{Människa-datorinteraktion}
\assignmenttype{Laboration}
\title{Användbarhet}
\author{Daniel Bosk\footnote{%
	Författarens e-post: \href{mailto:daniel.bosk@miun.se}{daniel.bosk@miun.se}.
}}
\date{\svnId}

\begin{document}
\maketitle
\thispagestyle{foot}
\tableofcontents


\section{Introduktion}
\label{sec:Introduktion}
Användbarhet\footnote{%
	\emph{Eng.} usability.
} är ett centralt begrepp inom interaktionsdesign och 
människa--datorinteraktion (MDI).
\citeauthor{Sharp2011idb} definierar användbarhet som
''\foreignlanguage{english}{ensuring that interactive products are easy to 
learn, effective to use, and enjoyable from the user's perspective.
It involves optimizing the interactions people have with interactive products 
to enable them to carry out their activities}'' \citep[sidan 19]{Sharp2011idb}.
Användbarhet, till skillnad från användarupplevelse, är av mer objektiv 
karaktär och kan på olika sätt mätas eller på annat sätt uppskattas.
Denna laboration handlar om dessa begrepp.


\section{Syfte}
\label{sec:Syfte}
Syftet med laborationen är att undersöka användbarheten hos något interaktivt 
system\footnote{%
  Eller, för att använda litteraturens terminologi, en  produkt.
}.
Ni ska också reflektera över sambandet mellan användbarhet och 
användarupplevelse, samt hur kognition och andra aspekter påverkar 
användbarheten och användarupplevelsen.
Detta för att öva på att se interaktiva system ur ett MDI-perspektiv.


\section{Läsanvisningar}
\label{sec:Lasanvisningar}
Du ska inför denna övning ha läst kapitlen 12--15, som handlar om utvärdering, 
i \emph{Interaction Design} \citep{Sharp2011idb}.



\section{Genomförande}
\label{sec:Genomforande}
Välj en typ av mobilt interaktivt system och undersök användbarheten hos detta.
Exempelvis
\begin{itemize}
  \item webbutiker på mobil eller datorplatta, där ni kan analysera hur det är 
    att genomföra en beställning och söka efter produkter;
  \item de fyra stora teleoperatörerna (4T) och de stora bankerna har släppt 
    appar inom området betala med mobilen, WyWallet\footnote{%
      Teleoperatörernas WyWallet, URL: \url{http://www.wywallet.se/}.
    } respektive Swish\footnote{%
      Bankernas Swish, URL: \url{http://www.getswish.se/}.
    }, där ni kan undersöka hur enkla dessa är att använda.
\end{itemize}
Men det är bara fantasin som sätter gränserna.

Detta mobila interaktiva system ska ni analysera ur ett MDI-perspektiv.
Ni ska alltså tillämpa teorin \cite{Sharp2011idb} på detta exempelsystem, 
använd teorin för att förklara vad som är bra respektive dåligt.


\section{Examination}
\label{sec:Examination}
Laborationen examineras genom muntlig presentation om 10 till 15 minuter.
Ni förbereder en presentation av ert arbete där ni beskriver vad som är bra och 
dåligt samt förklarar varför med hjälp av teorin.

Uppgiften får genomföras i grupper om två personer.
Båda gruppmedlemmarna ska kunna genomföra presentationen självständigt och 
svara på frågor utan den andres hjälp.
Vid presentationstillfället får ni veta vem av er som ska hålla presentationen.


\printbibliography
\end{document}
