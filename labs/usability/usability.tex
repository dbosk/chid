\documentclass[a4paper,logo,nocourse]{miunasgn}
\usepackage[utf8]{inputenc}
\usepackage[T1]{fontenc}
\usepackage[english,swedish]{babel}
\usepackage[hyphens]{url}
\usepackage{hyperref}
\usepackage{prettyref,varioref}
\usepackage[today,nofancy]{svninfo}
\usepackage[natbib,style=alphabetic,maxbibnames=99]{biblatex}
\addbibresource{literature.bib}
\usepackage[varioref,prettyref]{miunmisc}

\svnInfo $Id$
%\printanswers

\courseid{IG025G}
\course{Människa-datorinteraktion}
\assignmenttype{Laboration}
\title{Användbarhet}
\author{Daniel Bosk\footnote{%
  Detta verk är tillgängliggjort under licensen Creative Commons 
  Erkännande-DelaLika 2.5 Sverige (CC BY-SA 2.5 SE).
  För att se en sammanfattning och kopia av licenstexten besök URL 
  \url{http://creativecommons.org/licenses/by-sa/2.5/se/}.
}}
\date{\svnId}

\begin{document}
\maketitle
\thispagestyle{foot}
\tableofcontents


\section{Introduktion}
\label{sec:Introduktion}
Användbarhet\footnote{%
	\emph{Eng.} usability.
} är ett centralt begrepp inom interaktionsdesign och 
människa--datorinteraktion (MDI).
\citeauthor{Sharp2011idb} definierar användbarhet som
''\foreignlanguage{english}{ensuring that interactive products are easy to 
learn, effective to use, and enjoyable from the user's perspective.
It involves optimizing the interactions people have with interactive products 
to enable them to carry out their activities}'' \citep[sidan 19]{Sharp2011idb}.
Användbarhet, till skillnad från användarupplevelse, är av mer objektiv 
karaktär och kan på olika sätt mätas eller på annat sätt uppskattas.
Denna laboration handlar om dessa begrepp.


\section{Syfte}
\label{sec:Syfte}
Syftet med laborationen är att undersöka användbarheten hos något interaktivt 
system\footnote{%
  Eller, för att använda litteraturens terminologi, en  produkt.
}.
Ni ska också reflektera över sambandet mellan användbarhet och 
användarupplevelse, samt hur kognition och andra aspekter påverkar 
användbarheten och användarupplevelsen.
Detta för att öva på att se interaktiva system ur ett MDI-perspektiv.


\section{Läsanvisningar}
\label{sec:Lasanvisningar}
För att påbörja projektet ska du ha läst samtliga kapitel (1--15) i boken 
\citetitle{Sharp2011idb} \citep{Sharp2011idb} och \citetitle{Nielsen2013mu} 
\cite{Nielsen2013mu}.

Eftersom att ni i projektet kommer att arbeta med och göra undersökningar med 
människor ska ni även ha läst om Vetenskapsrådets forskningsetiska principer
\citep{VR2002fpi}.



\section{Genomförande}
\label{sec:Genomforande}
Välj en typ av interaktivt system och undersök användbarheten hos detta.
Exempelvis
\begin{itemize}
  \item webbutiker, där ni kan analysera hur det är att genomföra en 
    beställning och söka efter produkter;
  \item de fyra stora teleoperatörerna (4T) och de stora bankerna har släppt 
    appar inom området betala med mobilen, WyWallet\footnote{%
      Teleoperatörernas WyWallet, URL: \url{http://www.wywallet.se/}.
    } respektive Swish\footnote{%
      Bankernas Swish, URL: \url{http://www.getswish.se/}.
    }, där ni kan undersöka hur enkla dessa är att betala med;
  \item datorsystem, ni kan jämföra gränssnitten i olika system, exempelvis 
    Linux med Gnome 3 eller KDE Plasma\footnote{%
      Dessa system finns tillgängliga i sal L207 campus Sundsvall.
    }, Windows 8 med Metro etc.;
\end{itemize}
Men det är bara fantasin som sätter gränserna.

Detta interaktiva system ska ni analysera ur ett MDI-perspektiv.
Ni ska alltså tillämpa teorin på detta exempelsystem, använd teorin för att 
förklara vad som är bra och dåligt.


\section{Examination}
\label{sec:Examination}
Laborationen examineras genom muntlig presentation om 10 till 15 minuter.
Ni förbereder en presentation av ert arbete där ni beskriver vad som är bra och 
dåligt samt förklarar varför med hjälp av teorin.

Uppgiften får genomföras i grupper om två personer.
Båda gruppmedlemmarna ska kunna genomföra presentationen självständigt och 
svara på frågor utan den andres hjälp.
Vid presentationstillfället får ni veta vem av er som ska hålla presentationen.


\printbibliography
\end{document}
