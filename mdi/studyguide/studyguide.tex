\documentclass[a4paper,logo]{miunart}
\usepackage[utf8]{inputenc}
\usepackage[swedish]{babel}
\usepackage{prettyref,varioref}
\usepackage[hyphens]{url}
\usepackage{hyperref}
\usepackage{csquotes}
\usepackage{booktabs}
\usepackage[today,nofancy]{svninfo}
\usepackage[natbib,style=alphabetic,maxbibnames=99]{biblatex}
\addbibresource{literature.bib}
\usepackage[prettyref,varioref]{miunmisc}

\svnInfo $Id$

\title{Den kompletta studiehandledningen för kurserna\\
  DT157G/IG025G Människa--datorinteraktion och IU127G Interaktionsdesign}
\author{Daniel Bosk\footnote{%
  Detta verk är tillgängliggjort under licensen Creative Commons 
  Erkännande-DelaLika 2.5 Sverige (CC BY-SA 2.5 SE).
	För att se en sammanfattning och kopia av licenstexten besök URL 
	\url{http://creativecommons.org/licenses/by-sa/2.5/se/}.
}}
\date{\svnId}

\begin{document}
\maketitle
\tableofcontents


\section{Mål}

Interaktiva IT-system finns överallt liksom svårigheterna för vanliga människor 
att hantera dem.
Vad behöver vi veta för att människor ska lyckas att använda dessa system 
effektivt?
Hur kan vi förbättra sättet de utformas på?

Kursen syftar till att introducera dig till ämnet människa--maskininteraktion 
och interaktionsdesign med inriktning mot utveckling och utvärdering av 
digitala produkter och tjänster utifrån ett användarperspektiv.
Förståelsen om människans kognitiva förmågor vid utformning av digitala 
produkter och tjänster är förutsättningen för att skapa användbara interaktiva 
system.

Du ska efter avslutad kurs med godkänt resultat kunna
\begin{itemize}
    \input{project-aim.tex}
%  \item definiera grundläggande begrepp inom området interaktionsdesign,
%  \item beskriva och tillämpa användarcentrerade metoder för utveckling och 
%    utvärdering av interaktiva system,
%  \item analysera metoder utifrån olika teorier och modeller,
%  \item kombinera olika metoder eller delar därutav för att nå egna uppställda 
%    mål,
%  \item värdera sitt val av metod och föreslå eventuella förbättringar, och
%  \item ta hänsyn till etiska aspekter av forskningsarbete.
\end{itemize}


\section{Kursupplägg}

Boken som används som huvudlitteratur på kursen är \citetitle{Sharp2011idb} av 
\citet{Sharp2011idb}.
Boken behandlar interaktionsdesign (ID) och går lite djupare än enbart 
människa--datorinteraktion (MDI) traditionellt gör, MDI är ett delområde inom 
ID.
Boken ska läsas i sin helhet.
Boken \citetitle{Tullis2013mtu} av \citet{Tullis2013mtu} är ett bra komplement 
med fokus på undersökningsmetod.
Därutöver tillkommer en kort text om etik vid forskning med människor 
publicerad av \citet{VR2002fpi} och annat literatur (se nedan).

Kursens lärandemål examineras formativt genom ett antal seminarier och en 
laboration, därefter summativt med ett avslutande projekt.
Se lydelsen för respektive uppgift för detaljer.

\subsection{Schema}

Du finner en sammanställning av kursens schema i \prettyref{tbl:schema}.
Det är naturligtvis fritt att följa detta sånär som på datum för exmination av 
uppgifter och föreläsningarna.
Läsanvisningar för respektive moment följer i kommande avsnitt.
Undervisningen kräver att du följer dessa anvisningar.

\begin{table}
  \centering
  \begin{tabular}{cp{10cm}}
    \toprule
    \textbf{Kursvecka} & \textbf{Moment} \\
    \toprule
    1   & Kursstart/Introduktionsföreläsning \\
        & Föreläsning om konceptualisering \\
    \midrule
    2   & Föreläsning om kognition \\
        & Föreläsning om social interaktion \\
        & Laboration om användbarhet \\
    \midrule
    3   & Föreläsning om emotionell interaktion \\
        & Inledande övning om metod \\
        & Redovisning av laboration \\
    \midrule
    4   & Övning om datainsamling och -analys \\
        & Handledning \\
        & Övning om att fastställa mål \\
        & Handledning \\
        & Seminarium om designprinciper \\
    \midrule
    5   & Övning om design och prototypning \\
        & Handledning \\
        & Övning om utvärdering \\
        & Handledning \\
    \midrule
    6   & Handledning \\
        & Handledning \\
    \midrule
    7   & Handledning \\
        & Handledning \\
    \midrule
    8   & Handledning \\
    \midrule
    9   & Handledning \\
    \midrule
    10  & Redovisning av projekt \\
    \bottomrule
  \end{tabular}
  \caption{En sammanställning av kursens moment och när de kommer att 
    genomföras.
    Tiden är anpassad efter en studietakt om halvfart.
  }
  \label{tbl:schema}
\end{table}

\subsection{Introduktionsföreläsning}
\input{intro-lit.tex}

\subsection{Föreläsning om konceptualisering}
\input{concept-lit.tex}

\subsection{Föreläsning om kognition}
\input{cognition-lit.tex}

\subsection{Föreläsning om social interaktion}
\input{social-lit.tex}

\subsection{Föreläsning om emotionell interaktion}
\input{emotional-lit.tex}

\subsection{Laboration om användbarhet}
\input{lab-usability-lit.tex}

\subsection{Seminarium om designprinciper}
\input{sem-design-lit.tex}

\subsection{Inledande övning om metod}
\input{method-lit.tex}

\subsection{Övning om datainsamling och -analys}
\input{data-lit.tex}

\subsection{Övning om att fastställa mål}
\input{goals-lit.tex}

\subsection{Övning om design och prototypning}
\input{prototype-lit.tex}

\subsection{Övning om utvärdering}
\input{eval-lit.tex}

\subsection{Projektet}
\input{project-lit.tex}


\section{Examination}

Kursen examineras med en laboration, ett seminarium och ett projekt.
Laborationen och seminariet examineras muntligen och betygsätts med \emph{pass 
(P)} för godkänt betyg eller \emph{fail (F)} för underkänt betyg, detta 
motsvarar moment I101 i ladok.
Projektet betygsätts med betygen A--E för godkänt, F eller Fx för underkänt.
För detaljer, se lydelsen för respektive uppgift.
Projektet motsvarar moment P101 i ladok.

Alla inlämningar och rapporter ska vara i ''godkännbart'' skick; det vill säga, 
vara välformulerade, grammatiskt korrekta och utan stavfel, ha korrekta 
referenser, samt uppfylla samtliga krav i lydelsen.

Betyget Fx innebär möjlighet till komplettering.
I denna kurs ska komplettering göras inom en vecka från retur.
Utebliven komplettering resulterar att du hänvisas till nästa
rättningstillfälle.

Allt inlämnat material ska vara skapade av dig själv, eller vid gruppuppgifter, 
skapat av dig eller någon av dina gruppmedlemmar.
När du refererar och citerar andra verk måste korrekta källhänvisningar finnas 
och i fallet citering måste den citerade texten måste vara tydligt markerad.
Om plagiat finns i dokumentet riskerar du att stängas av från studier under 
bestämd tid, högst 6 månader på grund av disciplinförseelse.
Vid gruppuppgift riskerar alla gruppmedlemmar att hållas ansvariga
för disciplinförseelse om det av verket inte tydligt framgår vilka av
medlemmarna som ansvarat för de plagierade delarna.

Om samarbete sker utan att detta har stöd i instruktionen för examinationen
utgör det normalt en disciplinförseelse och studenterna riskerar att 
stängas av från studier under bestämd tid, högst sex månader.
Om inget annat anges i lydelsen är uppgifterna individuella.


\section{Vad händer om jag ej blir klar i tid?}
\label{sec:late}
Slutdatumena på denna kurs är av yttersta vikt.

De slutdatum som finns för dessa tillfällen är strikta.
Om du missar slutdatumet för ett tillfälle hänvisas du till nästa tillfälle.
Efter det tredje tillfället hänvisas du till tillfällena under nästkommande 
kursomgång.

För skriftliga inlämningsuppgifter gäller att dessa rättas en gång under
kursens gång, senast i samband med slutdatum för inlämning, därefter
ytterligare två gånger inom ett år.
Totalt erbjuds tre rättningstillfällen per år.
Därefter hänvisas du till nästa kursomgång.
Sent inlämnade uppgifter underkänns.
Möjlighet till förlängning kan ges, det ska dock finnas en mycket god anledning
och förlängning ska anhållas om i god tid.

Ingen handledning planeras efter kursens slut, det vill säga efter det sista
schemalagda handledningstillfället.
Du rekommenderas därför starkt att följa kursens schema, tänk på att ha
marginaler till slutdatum så att du hinner ta upp eventuellt problem vid ett
handledningstillfälle.
Vill du läsa om kursen kan du göra det genom att registrera om dig nästa gång
kursen ges.
Omregistrering på kurstillfälle sker i mån om plats, alla förstagångssökande
och reserver kommer att prioriteras.

Om du känner att du inte kommer att hinna bli klar med kursen är det därför
bättre att göra ett tidigt avbrott på kursen och söka om den inför nästa
kurstillfälle.
Tidigt avbrott kan registreras senast tre veckor från kursstart och då kommer
du att räknas som en förstagångssökande nästa gång du söker kursen.

\printbibliography
\end{document}
