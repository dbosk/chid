\documentclass[a4paper,logo]{miunart}
\usepackage[utf8]{inputenc}
\usepackage[swedish]{babel}
\usepackage{prettyref,varioref}
\usepackage[hyphens]{url}
\usepackage{hyperref}
\usepackage{csquotes}
\usepackage{booktabs}
\usepackage[today,nofancy]{svninfo}
\usepackage[natbib,style=authoryear-comp,maxbibnames=99]{biblatex}
\addbibresource{literature.bib}
\usepackage[prettyref,varioref]{miunmisc}

\svnInfo $Id$

\title{Den kompletta studiehandledningen för kursen\\
  DT126G Användbarhet för mobila enheter}
\author{Daniel Bosk\footnote{%
  Detta verk är tillgängliggjort under licensen Creative Commons 
  Erkännande-DelaLika 2.5 Sverige (CC BY-SA 2.5 SE).
	För att se en sammanfattning och kopia av licenstexten besök URL 
	\url{http://creativecommons.org/licenses/by-sa/2.5/se/}.
}}
\date{\svnId}

\begin{document}
\maketitle
\tableofcontents


\section{Mål}

Interaktiva mobilsystem finns överallt, liksom svårigheterna för vanliga 
människor att hantera dem.
Vad behöver vi veta för att människor ska lyckas att använda dessa system 
effektivt?
Hur kan vi förbättra sättet de utformas på?

Kursen syftar till att introducera dig till ämnet interaktionsdesign med 
inriktning mot utveckling och utvärdering av interaktiva mobila system.
Förståelsen om människans kognitiva förmågor vid utformning av dessa system är 
förutsättningen för att skapa användbara mobilsystem.

Du ska efter avslutad kurs med godkänt resultat kunna
\begin{itemize}
  \input{project-aim.tex}
%  \item definiera grundläggande begrepp inom området interaktionsdesign,
%  \item beskriva och tillämpa användarcentrerade metoder för utveckling och 
%    utvärdering av interaktiva system,
%  \item analysera metoder utifrån olika teorier och modeller,
%  \item kombinera olika metoder eller delar därutav för att nå egna uppställda 
%    mål,
%  \item värdera sitt val av metod och föreslå eventuella förbättringar, och
%  \item ta hänsyn till etiska aspekter av forskningsarbete.
\end{itemize}


\section{Kursupplägg}

För kursen används två böcker; \citetitle{Sharp2011idb} \cite{Sharp2011idb} och 
\citetitle{Nielsen2013mu} \cite{Nielsen2013mu}.
\citetitle{Sharp2011idb} behandlar interaktionsdesign (ID) och går lite djupare 
än enbart människa--datorinteraktion (MDI) traditionellt gör, MDI är ett 
delområde inom ID.
Boken går igenom grundläggande teori och metod.

\citetitle{Nielsen2013mu} fokuserar på mobilanvändbarhet.
Den sammanfattar resultaten från många års studier av användbarhet för mobila 
enheter.
Denna bok innehåller således mer konkreta riktlinjer för design av 
mobilapplikationer.

Därutöver tillkommer en kort text om etik vid forskning med människor 
publicerad av \citet{VR2002fpi} och annat literatur (se nedan).

Kursens lärandemål examineras formativt genom ett antal seminarier och en 
laboration, därefter summativt med ett avslutande projekt.
Se lydelsen för respektive uppgift för detaljer.

\subsection{Schema}

Du finner en sammanställning av kursens schema i \prettyref{tbl:schema}.
Det är naturligtvis fritt att följa detta sånär som på datum för exmination av 
uppgifter och föreläsningarna.
Läsanvisningar för respektive moment följer i kommande avsnitt.
Undervisningen kräver att du följer dessa anvisningar.

\begin{table}
  \centering
  \begin{tabular}{cp{10cm}}
    \toprule
    \textbf{Kursvecka} & \textbf{Moment} \\
    \toprule
    1   & Kursstart/Introduktionsföreläsning \\
    \midrule
    2
        & Föreläsning om konceptualisering \\
        & Föreläsning om kognition \\
    \midrule
    3
        & Föreläsning om social interaktion \\
        & Föreläsning om emotionell interaktion \\
    \midrule
    4
        & Seminarium om designprinciper \\
        & Inledande övning om metod \\
    \midrule
    5
        & Seminarium om mobilanvändbarhet \\
        & Övning om datainsamling och -analys \\
        & Handledning \\
    \midrule
    6
        & Övning om att fastställa mål \\
    \midrule
    7
        & Övning om design och prototypning \\
    \midrule
    8
        & Övning om utvärdering \\
%    \midrule
%    9
%        & Eget arbete \\
    \midrule
    10
        & Redovisning av projekt \\
    \bottomrule
  \end{tabular}
  \caption{En sammanställning av kursens moment och när de kommer att 
    genomföras.
    Tiden är anpassad efter en studietakt om halvfart.
  }
  \label{tbl:schema}
\end{table}

\subsection{Introduktionsföreläsning}
\input{intro-lit.tex}

\subsection{Föreläsning om konceptualisering}
\input{concept-lit.tex}

\subsection{Föreläsning om kognition}
\input{cognition-lit.tex}

\subsection{Föreläsning om social interaktion}
\input{social-lit.tex}

\subsection{Föreläsning om emotionell interaktion}
\input{emotional-lit.tex}

\subsection{Seminarium om designprinciper}
\input{sem-design-lit.tex}

\subsection{Seminarium om mobilanvändbarhet}
\input{sem-usability-lit.tex}

\subsection{Inledande övning om metod}
\input{method-lit.tex}

\subsection{Övning om datainsamling och -analys}
\input{data-lit.tex}

\subsection{Övning om att fastställa mål}
\input{goals-lit.tex}

\subsection{Övning om design och prototypning}
\input{prototype-lit.tex}

\subsection{Övning om utvärdering}
\input{eval-lit.tex}

\subsection{Projektet}
\input{project-lit.tex}


\section{Examination}

Kursen examineras med två seminarier och ett projekt.
Seminariena examineras muntligen och betygsätts med \emph{pass (P)} för godkänt 
betyg eller \emph{fail (F)} för underkänt betyg.
Projektet betygsätts med betygen A--E för godkänt, F eller Fx för underkänt.
För detaljer, se lydelsen för respektive uppgift.

Alla inlämningar och rapporter ska vara i ''godkännbart'' skick; det vill säga, 
vara välformulerade, grammatiskt korrekta och utan stavfel, ha korrekta 
referenser, samt uppfylla samtliga krav i lydelsen.

Betyget Fx innebär möjlighet till komplettering.
Denna komplettering ska göras inom en vecka.
Utebliven komplettering resulterar i F och du hänvisas till nästa tillfälle.

Allt inlämnat material ska vara skapade av dig själv, eller vid gruppuppgifter, 
skapat av dig eller någon av dina gruppmedlemmar.
När du refererar och citerar andra verk måste korrekta källhänvisningar finnas 
och i fallet citering måste den citerade texten måste vara tydligt markerad.
Om plagiat finns i dokumentet riskerar du att stängas av från studier under 
bestämd tid, högst 6 månader på grund av disciplinförseelse.
Vid gruppuppgift riskerar alla gruppmedlemmar att hållas ansvariga
för disciplinförseelse om det av verket inte tydligt framgår vilka av
medlemmarna som ansvarat för de plagierade delarna.

Om samarbete sker utan att detta har stöd i instruktionen för examinationen
utgör det normalt en disciplinförseelse och studenterna riskerar att
stängas av från studier under bestämd tid, högst sex månader.


\section{Vad händer om jag ej blir klar i tid?}
\label{sec:late}
%Du måste ha genomfört introduktionsuppgiften inom dess givna slutdatum, om du 
%inte gör detta kommer du att avregistreras från kursen och din plats kommer 
%att ställas till förfogande för andra sökande.
%
För examinationen på kursen kommer det för redovisningar och seminarier att ges 
ett tillfälle under kursens gång.
Därefter ges ytterligare två redovisningstillfällen, dessa förläggs inom ett 
år.
Alla dessa tillfällen kommer att finnas i kursens schema (i studentportalen).

De slutdatum som finns för dessa tillfällen är strikta.
Om du missar slutdatumet för ett tillfälle hänvisas du till nästa 
redovisningstillfälle.
Efter det tredje redovisningstillfället hänvisas du till tillfällena under 
nästkommande kursomgång.

För skriftliga inlämningsuppgifter gäller att dessa rättas en gång under 
kursens gång, senast i samband med slutdatum för inlämning, därefter 
ytterligare två gånger i de kommande omtentamensperioderna.
Totalt erbjuds tre försök per år.
Därefter hänvisas du till nästa kursomgång.
Sent inlämnade uppgifter underkänns och hänvisas till nästa tillfälle.
Möjlighet till förlängning kan ges, det ska dock finnas en mycket god anledning 
och förlängning ska anhållas om i god tid.

Ingen handledning planeras efter kursens slut, det vill säga efter det sista 
schemalagda handledningstillfället.
Om du inte hinner bli klar med uppgifterna inom kursens tidsramar och du vill 
vara garanterad handledning av lärare krävs att du omregistrerar dig på nästa 
kurstillfälle.
Omregistrering på kurstillfälle sker i mån om plats, alla förstagångssökande 
och reserver kommer att prioriteras.

Om du vid kursslut har majoriteten av kursens moment kvar att göra hänvisas du 
direkt till nästa kursomgång, då krävs omregistrering.
Huruvida din prestation är tillräcklig för att enbart komplettera eller om 
omregistrering krävs bedöms av ansvarig lärare.

Om du känner att du inte kommer att hinna bli klar med kursen är det därför 
bättre att göra ett tidigt avbrott på kursen och söka om den inför nästa 
kurstillfälle.
Tidigt avbrott kan registreras senast tre veckor från kursstart och då kommer 
du att räknas som en förstagångssökande nästa gång du söker kursen.


\printbibliography
\end{document}
