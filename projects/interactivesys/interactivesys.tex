\documentclass[a4paper,logo,nocourse]{miunasgn}
\usepackage[utf8]{inputenc}
\usepackage[T1]{fontenc}
\usepackage[english,swedish]{babel}
\usepackage[hyphens]{url}
\usepackage{hyperref}
\usepackage{prettyref,varioref}
\usepackage[today,nofancy]{svninfo}
\usepackage[natbib,style=alphabetic,maxbibnames=99]{biblatex}
\addbibresource{literature.bib}
\usepackage[varioref,prettyref]{miunmisc}

\svnInfo $Id$
%\printanswers

\courseid{IG025G}
\course{Människa-datorinteraktion}
\assignmenttype{Projekt}
\title{Utveckling av ett användargränssnitt}
\author{Daniel Bosk\footnote{%
  Detta verk är tillgängliggjort under licensen Creative Commons 
  Erkännande-DelaLika 2.5 Sverige (CC BY-SA 2.5 SE).
  För att se en sammanfattning och kopia av licenstexten besök URL 
  \url{http://creativecommons.org/licenses/by-sa/2.5/se/}.
}}
\date{\svnId}

\begin{document}
\maketitle
\thispagestyle{foot}
\tableofcontents


\section{Introduktion}
\label{sec:Introduktion}
Interaktionsdesign fokuserar på utvecklingen av system som ska stödja 
användaren i dess vardagliga kommunikation och interaktion, oavsett om det är 
kommunikation mellan människa--dator eller människa--människa via dator.
Det resulterande systemet ska vara optimerat för användbarhet och 
användarupplevelse.

Genom åren av forskning inom området människa--datorinteraktion har det tagits 
fram olika metoder för att underlätta utvecklingen av användargränssnitt.
De metoder som används är användarcentrerade och fokuserar på användarens mål 
med användningen av systemet.
Det har tagits fram generella mål för användbarhet och användarupplevelse, 
olika designprinciper och kunskap från andra forskningsområden, såsom kognitiv 
psykologi, har integrerats inom området.


\section{Syfte}
\label{sec:Syfte}
Syftet med detta projekt är att ni ska öva på arbetssätten inom 
människa--datorinteraktion, och interaktionsdesign generellt.
Det vill säga arbeta användarcentrerat från ursprungsidé till ett färdigt 
designförslag.

Därutöver ska projektet examinera att ni kan:
\begin{itemize}
    \item definiera grundläggande begrepp inom området interaktionsdesign.
\item beskriva och tillämpa användarcentrerade metoder för utveckling och 
utvärdering av interaktiva system.
\item analysera metoder utifrån olika teorier och modeller.
\item kombinera olika metoder eller delar därutav för att nå egna uppställda 
mål.
\item värdera sitt val av metod och föreslå eventuella förbättringar.
\item ta hänsyn till etiska aspekter av forskningsarbete.

\end{itemize}


\section{Läsanvisningar}
\label{sec:Lasanvisningar}
Du ska inför denna övning ha läst kapitlen 12--15, som handlar om utvärdering, 
i \emph{Interaction Design} \citep{Sharp2011idb}.



\section{Genomförande}
\label{sec:Genomforande}
Ni ska ta fram en design för ett interaktivt gränssnitt.
Målgruppen får ni själva, inom designteamet, välja.

Projektets arbetsgång kommer översiktligt att följa dessa steg:
\begin{enumerate}
	\item målanalys,
  \item ta fram designalternativ,
	\item utvärdera med målgruppen, och till sist
	\item ge ett slutgiltigt förslag.
\end{enumerate}
Detaljerna för varje steg, eventuella mellansteg och iteratioer, lämnas till 
litteraturen.
Anledningen är att ni ska få välja de metoder i litteraturen som ni finner bäst 
lämpade för just ert projekt\footnote{%
	På så vis kommer det sannolikt att bli intressantare som lyssnare vid 
	presentationerna om inte alla gjort exakt samma sak.
}.


\section{Examination}
\label{sec:Examination}
Projektet får genomföras i grupper om tre till fyra personer.
Det redovisas både muntligt och skriftligt.

Projektet redovisas först muntligen vid ett presentationstillfälle inför 
helklass.
Presentationens omfattning bör vara mellan 15 och 20 minuter.
Samtliga gruppmedlemmar ska kunna genomföra den muntliga presentationen av 
projektet utan övriga medlemmars hjälp, se därför till att samtliga medlemmar 
är väl insatta i alla delar av projektet -- arbeta tillsammans, dela inte upp 
arbetet!
Examinator utser vid presentationstillfället vem av medlemmarna som 
presenterar, kom därför väl förberedda.

Vid presentationstillfället får ni frågor, kommentarer och inspiration från 
övriga presentationer som kan hjälpa er till ett djupare diskussionsavsnitt för 
rapporten.
Projektet redovisas därefter med en skriftlig rapport enligt utformande av en 
akademisk rapport; det vill säga med huvudavsnitten introduktion, teori, metod, 
resultat, analys och diskussion.
(Givetvis även korrekta referenser.)
Observera att det i rapporten tydligt ska framgå vilka avväganden och 
hänsynstaganden till de forskningsetiska principerna ni gjort.
En riktlinje för rapportens omfattning är cirka 10 till 20 sidor exklusive 
bilagor, men det är kvaliteten på innehållet och inte antalet sidor som är 
väsentligt.

Följande kriterier gäller för för betygsättning:
\begin{description}
  \item[E] Rapporten uppfyller samtliga krav i lydelsen (men författarna har 
    följt ''minsta motståndets lag'').

    %Exempelvis om en enkätundersökning utgör den enda datainsamlingen 
    %kvalificerar detta inte för ett E, det utgör inte ens en iteration.

  \item[D] Uppfyller samtliga tidigare, men inte hela vägen till C.

  \item[C] Uppfyller samtliga tidigare betygsteg.
    Det är ett väl genomarbetat projekt med flera användarundersökningar och 
    användbarhetstester.
    Teorin är väl kopplad till resultaten, d.v.s.\ resultaten förklaras 
    i analysen med hjälp av teorin och diskuteras i diskussionen.

  \item[B] Uppfyller samtliga tidigare, men inte hela vägen till A.

  \item[A] Uppfyller samtliga tidigare betygsteg.
    Det finns stark anknytning till forskningslitteratur eller annan 
    anmärkningsvärd anledning för högsta betyg, exempelvis flertalet 
    iterationer i processen.
\end{description}


\printbibliography
\end{document}
