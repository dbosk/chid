% $Id$
\documentclass[a4paper,logo,nocourse]{miunasgn}
\usepackage[utf8]{inputenc}
\usepackage[T1]{fontenc}
\usepackage[english,swedish]{babel}
\usepackage[hyphens]{url}
\usepackage{hyperref}
\usepackage{prettyref,varioref}
\usepackage[today,nofancy]{svninfo}
\usepackage[natbib,style=alphabetic,maxbibnames=99]{biblatex}
\addbibresource{literature.bib}
\usepackage[varioref,prettyref]{miunmisc}

\svnInfo $Id$
%\printanswers

\courseid{IG025G}
\course{Människa-datorinteraktion}
\assignmenttype{Seminarium}
\title{Olika gränssnittsdesign}
\author{Daniel Bosk\footnote{%
	E-post: \href{mailto:daniel.bosk@miun.se}{daniel.bosk@miun.se}.
  Detta verk är tillgängliggjort under licensen Creative Commons 
  Erkännande-DelaLika 2.5 Sverige (CC BY-SA 2.5 SE).
	För att se en sammanfattning och kopia av licenstexten besök URL 
	\url{http://creativecommons.org/licenses/by-sa/2.5/se/}.
}}
\date{\svnId}

\begin{document}
\maketitle
\thispagestyle{foot}
\tableofcontents


\section{Introduktion}
\label{sec:Introduktion}
Det finns många olika användargränssnitt, flera av vilka som löser samma 
problem för användaren.
Detta kan naturligtvis göras olika väl för olika användare.
I denna uppgift ska vi titta närmare på tre olika gränssnitt för 
mobiltelefoner; nämligen Android, Meego och iOS.


\section{Syfte}
\label{sec:Syfte}
Syftet med uppgiften är att diskutera de olika aspekterna av 
användargränssnitten hos Android, Meego och iOS, att få en insikt i hur de är 
designade och varför, samt hur skillnaderna påverkar användbarheten hos 
systemen.


\section{Läsanvisningar}
\label{sec:Lasanvisningar}
Du ska inför denna övning ha läst kapitlen 12--15, som handlar om utvärdering, 
i \emph{Interaction Design} \citep{Sharp2011idb}.



\section{Genomförande}
\label{sec:Genomforande}
Läs igenom dokumenten och reflektera över din erfarenhet med system av denna 
typ, särskilt om du har erfarenhet av något utav dessa system.
Koppla detta till det du läst i \citetitle{Sharp2011idb} \citep{Sharp2011idb}.
Anteckna dina tankar och ta med dem till seminariet.
Samtliga deltagare ska presentera en fråga relaterad till innehållet som de 
vill diskutera under seminariet.


\section{Examination}
\label{sec:Examination}
\emph{Aktivt} deltagande vid seminariet fordras för godkänt resultat.
Seminariet kommer att hålla följande struktur:
\begin{itemize}
  \item Öppen diskussion om samtliga deltagares reflektioner efter att ha läst 
    \cite{Nokia2011n9u}, \cite{Android2012d} och \cite{Apple2012hig}.

  \item Inspirationsfrågor för fördjupande diskussion:
    \begin{itemize}
      \item Vad är bra?
      \item Vad är dåligt?
      \item Varför har de valt att göra så här?
      \item Finns exempel på appar som följer respektive inte följer 
        riktlinjerna?
    \end{itemize}

  \item Hur förhåller sig dessa riktlinjer till teorin i litteraturen 
    \citep{Sharp2011idb}?
\end{itemize}


\printbibliography
\end{document}
