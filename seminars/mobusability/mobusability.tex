% $Id$
\documentclass[a4paper,logo,nocourse]{miunasgn}
\usepackage[utf8]{inputenc}
\usepackage[T1]{fontenc}
\usepackage[english,swedish]{babel}
\usepackage[hyphens]{url}
\usepackage{hyperref}
\usepackage{prettyref,varioref}
\usepackage[today,nofancy]{svninfo}
\usepackage[natbib,style=authoryear-comp,maxbibnames=99]{biblatex}
\addbibresource{literature.bib}
\usepackage[varioref,prettyref]{miunmisc}

\svnInfo $Id$
%\printanswers

\courseid{IG025G}
\course{Människa-datorinteraktion}
\assignmenttype{Seminarium}
\title{Olika gränssnittsdesign}
\author{Daniel Bosk\footnote{%
	E-post: \href{mailto:daniel.bosk@miun.se}{daniel.bosk@miun.se}.
  Detta verk är tillgängliggjort under licensen Creative Commons 
  Erkännande-DelaLika 2.5 Sverige (CC BY-SA 2.5 SE).
	För att se en sammanfattning och kopia av licenstexten besök URL 
	\url{http://creativecommons.org/licenses/by-sa/2.5/se/}.
}}
\date{\svnId}

\begin{document}
\maketitle
\thispagestyle{foot}
\tableofcontents


\section{Introduktion}
\label{sec:Introduktion}
Det finns många ansatser till interaktiva mobila gränssnitt.
Vi har redan sett att de olika mobila operativsystemen har riktlinjer för hur 
gränssnitten ska struktureras \cite{Nokia2011n9u,Android2012d,Apple2012hig}, 
men det finns fortfarande viss frihet för appdesign.
Under detta seminarium ska vi diskutera hur denna frihet bör användas, vad 
designern bör tänka på.


\section{Syfte}
\label{sec:Syfte}
Syftet med uppgiften är att ta del av resultaten från några omfattande studier 
av användbarhet för mobila enheter, diskutera några konkreta fall och vad en 
appdesigner bör tänka på.


\section{Läsanvisningar}
\label{sec:Lasanvisningar}
För att påbörja projektet ska du ha läst samtliga kapitel (1--15) i boken 
\citetitle{Sharp2011idb} \citep{Sharp2011idb} och \citetitle{Nielsen2013mu} 
\cite{Nielsen2013mu}.

Eftersom att ni i projektet kommer att arbeta med och göra undersökningar med 
människor ska ni även ha läst om Vetenskapsrådets forskningsetiska principer
\citep{VR2002fpi}.



\section{Genomförande}
\label{sec:Genomforande}
Läs igenom litteraturen, stryk under eller anteckna medan du läser.
Anteckna dina tankar och ta med dem till seminariet.
Samtliga deltagare ska presentera en fråga relaterad till innehållet som de 
vill diskutera under seminariet.


\section{Examination}
\label{sec:Examination}
\emph{Aktivt} deltagande vid seminariet fordras för godkänt resultat.
Seminariet kommer att hålla följande struktur:
\begin{itemize}
  \item Öppen diskussion om samtliga deltagares reflektioner efter att ha läst 
    boken \cite{Nielsen2013mu}.

  \item Vad är bokens huvudpunkter?

  \item Vad säger er egen erfarenhet?

\end{itemize}


\printbibliography
\end{document}
